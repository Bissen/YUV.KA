
\documentclass{article}
\usepackage[utf8]{inputenc}
\usepackage{amsmath}
\usepackage{amssymb}
\begin{document}
\title{Projekt der Softwareentwicklung 2011/2012}
\author{Max Wagner, Patrick Gemander, Sebastian Ullrich, Michael Vollmer \\ Robert Hangu, Daniel Lebert}
\maketitle
Inputvideos werden unkomprimiert sein, eine YUV-Datei.


General:
    \begin{itemize}
        \item (multithreaded)
        \item (Für Linux: GTK-Bindings)
    \end{itemize}
Input/Output:
    \begin{itemize}

        \item Video/EncodedVideo-Input (EncodedVideo ist Subklasse von Video, die ihre Logs mitführt)
        \item (Perlin)-Noise (+/- Intensity)
        \item Standbild und feste Farbe als Input
        \item Datei als .yuv-Datei abspeichern
    \end{itemize}
Manipulation:
    \begin{itemize}
        \item Farbkanäle trennen ($Video \rightarrow Video^3$)
        \item Videos mergen ($Video^n \rightarrow Video$)
            \begin{itemize}
                \item Additiv zusammenrechnen (zB. RGB mergen)
                \item Gewichtet mitteln
            \end{itemize}
        \item Blur (Gaussian, Linear, etc.?)
        \item Contrast, Saturation, Brightness
        \item Invert
        \item Quantisierer (Reduktion der Farbpalette)
        \item Video um $n$ Frames verschieben.
    \end{itemize}
Analyse:
    \begin{itemize}
        \item Öffnen bei Klick eigenes Fenster
        \item Diagrammansicht über Zeit (1 - n Inputs Video/EncodedVideo)
            \begin{itemize}
                \item Ein Input über Drop-down als Referenzvideo festlegen
                \item Helligkeitsverlauf
                \item Diff von Frame n $\rightarrow$ (n+1)
                \item Graphen:
                    \begin{itemize}
                        \item farbig abgegrenzt, ein Farbton pro Graphtyp
                        \item Funktionen über die Zeit
                        \item alte Frames verschwinden links aus dem Bild statt Stauchung
                        \item Zurückscrollbar
                        \item y-Skalierungen separat, dynamisch
                        \item da höchstens 4 verschiedene Skalen zwei Skalendarstellungen links, zwei rechts vom Graph
                        \item x-Skalierung fest
                    \end{itemize}
                \item Zu analysierendes Video über weiteres Drop-down auswählbar, in Instanz dann Graphtyp auswählbar
                \item Graphen ($Video \rightarrow \mathbb{R}$)
                    \begin{itemize}
                        \item I/P/B-Frames
                    \end{itemize}
                \item Graphen ($Video \rightarrow Referenzvideo \rightarrow \mathbb{R}$)
                    \begin{itemize}
                        \item Differenz Log-Entscheidungen
                        \item Differenz Pixelfarben
                        \item Anzahl Artefakte anhand der Referenz
                    \end{itemize}
            \end{itemize}
        \item Histogramm
        \item Diff
        \item Pseudonoise ratio
        \item Video-Outputs
            \begin{itemize}
                \item Simples Video-Output mit eigenem Output-Knoten ($\rightarrow Debugging$)
                \item Artefakte overlayen
            \end{itemize}
    \end{itemize}
GUI (Skizzen via Balsamiq):
    \begin{itemize}
        \item Größtenteils graphbasiert. (Nodes, etc.)
        \item Analyse und Manipulation in einer GraphView, in der man auch Analyseknoten nach Filter schalten kann
        \item Einfache Kantendarstellung als Bezierkurven, Drag\&Drop von Eingang auf Ausgang
    \end{itemize}


\end{document}
