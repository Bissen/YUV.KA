\documentclass{article}
\usepackage[utf8]{inputenc}
\usepackage{tabularx}

\begin{document}

\section{Qualitätsbestimmung}

%\begin{table}[htbp]
%\begin{tabularx}{1.2\textwidth}{|l|X|X|X|X|}
\begin{tabular}{@{\extracolsep{\fill}} |l|c|c|c|c|}
\hline
Produktqualität &  Sehr gut & Gut & Normal & Nicht relevant \\ \hline
\textbf{Funktionalität} &  &  &  &  \\ \hline
Angemessenheit  &  & X &  &  \\ \hline
Richtigkeit  &  & X &  &  \\ \hline
Interoperabilität  &  &  & X &  \\ \hline
Ordnungsmäßigkeit  &  &  & X &  \\ \hline
Sicherheit  &  &  & X &  \\ \hline
\textbf{Zuverlässigkeit} &  &  &  &  \\ \hline
Reife  &  &  & X &  \\ \hline
Fehlertoleranz  &  &  & X &  \\ \hline
Wiederherstellbarkeit  &  &  & X &  \\ \hline
\textbf{Benutzbarkeit} &  &  &  &  \\ \hline
Verständlichkeit  & X &  &  &  \\ \hline
Erlernbarkeit  & X &  &  &  \\ \hline
Bedienbarkeit & X &  &  &  \\ \hline
\textbf{Effizienz} &  &  &  &  \\ \hline
Zeitverhalten  &  & X &  &  \\ \hline
Verbrauchsverhalten  &  &  & X &  \\ \hline
\textbf{Änderbarkeit} &  &  &  &  \\ \hline
Analysierbarkeit &  & X &  &  \\ \hline
Modifizierbarkeit & X &  &  &  \\ \hline
Stabilität &  & X &  &  \\ \hline
Prüfbarkeit &  & X &  &  \\ \hline
\textbf{Übertragbarkeit} &  &  &  &  \\ \hline
Anpassbarkeit &  &  & X &  \\ \hline
Installierbarkeit &  &  & X &  \\ \hline
Konformität  &  &  & X &  \\ \hline
Austauschbarkeit  &  &  & X &  \\ \hline
\end{tabular}
%\end{tabularx}
%\end{table}
%\bigskip
\paragraph{}
Es wird Wert auf die Benutzbarkeit und Modifizierbarkeit gelegt sowie auf die Angemessenheit und Richtigkeit.

\end{document}
