\section{Testfälle}

\textbf{Folgende Funktionssequenzen sind zu überprüfen:}
\begin{itemize}
	\item\textbf{/T10/} Eingabeknoten erstellen, verschieben, verändern, löschen
	\item\textbf{/T20/} Manipulationsknoten erstellen, verschieben, verändern, löschen
	\item\textbf{/T30/} Endknoten erstellen, verschieben, öffnen, schließen, löschen
	\item\textbf{/T40/} Erstellen, Verschieben und Löschen von Kanten zwischen Knoten
	\item\textbf{/T50/} Pipeline erstellen, speichern, laden
	\item\textbf{/T60/} Wiedergabe starten, stoppen, Abspielgeschwindigkeit ändern, fortsetzen, zurücksetzen
	\item\textbf{/T70/} YUV-Datei einlesen, manipulieren, wiedergeben, speichern
	\item\textbf{/T80/} Diagrammknoten öffnen, Referenzvideo festlegen, Graph hinzufügen, Graphtyp ändern, Graph entfernen, Diagramm schließen, Knoten deaktivieren, Knoten 
				reaktivieren
\end{itemize}

~\\

\textbf{Folgende Datenkonsistenzen sind einzuhalten:}
\begin{itemize}
	\item\textbf{/T100/} Wiedergabe ist nur möglich, wenn mindests ein Endknoten geöffnet ist.
	\item\textbf{/T110/} Wiedergabe ist nur möglich, wenn alle mit der Pipeline verbundenen Eingabeknoten eine Quelle besitzen.
	\item\textbf{/T120/} Das Speichern einer YUV-Datei nach einem Knoten ist nur möglich, wenn dieser Knoten ein gültiges Video als Eingabe besitzt oder ein Eingabeknoten mit 
				gültiger Quelle ist.
	\item\textbf{/T130/} Die Pipeline kann nicht verändert werden, während die Wiedergabe läuft oder pausiert ist. Ausnahme hierbei sind die Optionen der Manipulationsknoten.
\end{itemize}
