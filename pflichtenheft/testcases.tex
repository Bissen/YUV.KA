\section{Testfälle}

Folgende Funktionssequenzen sind zu überprüfen:
\begin{itemize}
	\item /T01/
		\begin{itemize}
			\item Eingabeknoten erstellen, verschieben, verändern, löschen
		\end{itemize}
	\item /T02/
		\begin{itemize}
			\item Manipulationsknoten erstellen, verschieben, verändern, löschen
		\end{itemize}
	\item /T03/
		\begin{itemize}
			\item Endknoten erstellen, verschieben, öffnen, schließen, löschen
		\end{itemize}
	\item /T04/
		\begin{itemize}
			\item Erstellen, Verschieben und Löschen von Kanten zwischen Knoten
		\end{itemize}
	\item /T05/
		\begin{itemize}
			\item Graphen erstellen, speichern, laden
		\end{itemize}
	\item /T06/
		\begin{itemize}
			\item Wiedergabe starten, stoppen, Abspielgeschwindigkeit ändern, fortsetzen, zurücksetzen
		\end{itemize}
	\item /T07/
		\begin{itemize}
			\item YUV-Datei einlesen, manipulieren, wiedergeben, speichern
		\end{itemize}
	\item /T08/
		\begin{itemize}
			\item Diagrammknoten öffnen, Referenzvideo festlegen, Graph hinzufügen, Graphtyp ändern, Graph entfernen, Diagramm schließen, Knoten deaktivieren, Knoten 
			reaktivieren
		\end{itemize}
\end{itemize}

\newpage
Folgende Datenkonsistenzen sind einzuhalten:
\begin{itemize}
	\item /T20/
		\begin{itemize}
			\item Wiedergabe ist nur möglich, wenn mindests ein Endknoten geöffnet ist
		\end{itemize}
	\item /T21/
		\begin{itemize}
			\item Wiedergabe ist nur möglich, wenn alle mit dem Graphen verbundenen Eingabeknoten eine Quelle besitzen
		\end{itemize}
	\item /T22/
		\begin{itemize}
			\item Das Speichern einer YUV-Datei nach einem Knoten ist nur möglich, wenn dieser Knoten ein gültiges Video als Eingabe besitzt oder ein Eingabeknoten mit 
			gültiger Quelle ist
		\end{itemize}
	\item /T23/
		\begin{itemize}
			\item Der Graph kann nicht verändert werden, während die Wiedergabe läuft oder pausiert ist. Ausnahme hierbei sind die Optionen der Manipulationsknoten
		\end{itemize}
	\end{itemize}
