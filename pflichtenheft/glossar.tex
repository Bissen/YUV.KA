\section{Glossar}

\begin{itemize}
    \item[(Video-)Artefakte] Darstellungsfehler, welche sich dadurch zeigen, dass das encodete Video sichtlich stark von dem Referenzvideo abweicht.
    \item[Asynchrone UI] Asynchron gestaltete Oberfläche, die bei Ladeoperationenweiterhin auf Interaktion reagiert.
    \item[Differenzvideo] Videostream welcher aus den Daten besteht die bei der Pixelweisen Subtraktion der Farbwerte zweier Videos besteht.
    \item[Farbkanal] Ein Farbbild besteht in der Regel aus 3 Farbkanälen: Rot, grün und blau. Diese Kanäle entsprechen dem jeweiligen Teil des Bildes, der nur aus dieser Farbe besteht.
    \item[Frame] Einzelbild eines Videos, bei H.264 unterteilt in \emph{Macroblöcke}
    \item[Graph] Allgemeine Datenstruktur, bestehend aus ``Knoten'', welche durch Kanten verbunden sind.
    \item[H.264] Weit verbreiteter Video-Codec und Ziel-Codec der zu testenden Encoder
    \item[Histogramm] Diagrammart ähnlich einem Balkendiagramm, welche aber auch den Flächeninhalt eines Balken korrekt skaliert darstellt.
    \item[Interface, UI] Benutzeroberfläsche. Man unterscheidet zwischen Konsolenoberfläche und graphischer Oberfläche.
    \item[Inter-Frame] Videoframe, dessen Daten aus einem oder mehreren benachbarten Frames berechnet werden.
    \item[Intra-Frame] Videoframe, dessen Daten unabhängig von denen anderer Frames gepspeichert sind. Vergleiche \emph{Inter-Frame}.
    \item[Macroblock] Bei H.264 16x16 Pixel große Blöcke, in die jeder Frame aufgeteilt wird. Im Gegensatz zu älteren Codecs kann bei H.264 für jeden Macroblock eine unterschiedliche \emph{Inter/Intra-Frame}-Entscheidung getroffen werden. 
    \item[Move Vector] Vektor, um den jeder Referenzframe eines \emph{Inter-Frames} zusätzlich verschoben kann. Ermöglicht eine effiziente Kodierung von verschiebenden Bewegungen.
    \item[Multithreading] Programmiertechnik in der die zu verrichtende Arbeit auf mehrere Prozessfäden verteilt wird. Dies ermöglicht Parallelismus und die optimale Auslachtung von Multikern-Systemen
    \item[Noise] Ein randomisierten Signal. Allgemein als ``Bildrauschen'' zu verstehen.
    \item[Overlay] Ein transparent über ein anderes Objekt gezeichnetes Objekt.
    \item[Pipeline] Eine Einheit bestehend aus Arbeitsabläufen, die zum Fließbandbetrieb hintereinander geschaltet werden können.
    \item[Pseudo-Noise-Radio] TODO %seriously guys - what is this?
    \item[Resolution] Auflösung
    \item[Tabs] ``Register'': ähnlich der Register eines Aktenschrankes ist hier das dazugehörige Benutzeroberflächenelement gemeint.
    \item[Undo/Redo] ``Aktion zurücknehmen/Wiederholen''
    \item[Videoencoder] Programm, welches rohe Videodaten als Eingabe annimmt und in ein einem bestimmten Video-Codec entsprechendes Format bringt.
    \item[YUV] Eine Familie von Farbpaletten, welche die Sehart des menschlichen Auges in Betracht nehmen.
\end{itemize}
