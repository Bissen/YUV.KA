\section{Glossar}

\begin{itemize}
    \item Videoencoder - Programm welches rohe Videodaten als Eingabe annimmt, und in ein einem bestimmten Video-Codec entsprechendes Format bringt.
    \item Noise - Ein randomisierten Signal. Allgemein als ``Bildrauschen'' zu verstehen.
    \item Pipeline - Eine Einheit bestehend aus Arbeitsabläufen, die zum Fließbandbetrieb hintereinander geschaltet werden können.
    \item Graph - Allgemeine Datenstruktur, bestehend aus ``Knoten'', welche durch Kanten verbunden sind.
    \item Overlay - Ein transparent über ein anderes Objekt gezeichnetes Objekt.
    \item Undo/Redo - ``Aktion zurücknehmen/Wiederholen''
    \item Interface, UI - Benutzeroberfläsche. Man unterscheidet zwischen Konsolenoberfläche und graphischer Oberfläche.
    \item Tabs - ``Register'': ähnlich der Register eines Aktenschrankes ist hier das dazugehörige Benutzeroberflächenelement gemeint.
    \item Resolution - Auflösung
    \item Multithreading - Programmiertechnik in der die zu verrichtende Arbeit auf mehrere Prozessfäden verteilt wird. Dies ermöglicht Parallelismus und die optimale Auslachtung von Multikern-Systemen
    \item Asynchrone, nichtblockende UI - Asynchron gestaltete Oberfläche, die bei Ladeoperationenweiterhin auf Interaktion reagiert.
    \item YUV - Eine Familie von Farbpaletten welche die Sehart des menschlichen Auges in Betracht nehmen.
    \item Farbkanal - Ein Farbbild besteht in der Regel aus 3 Farbkanälen: Rot, grün und blau. Diese Kanäle entsprechen dem jeweiligen Teil des Bildes, der nur aus dieser Farbe besteht.
    \item Differenzvideo - Videostream welcher aus den Daten besteht die bei der Pixelweisen Subtraktion der Farbwerte zweier Videos besteht.
    \item I/P/B-Frame - TODO: verschiedene Arten von Videoframe. H.264-spezifisch
    \item Histogramm - Diagrammart ähnlich einem Balkendiagramm, welche aber auch den Flächeninhalt eines Balken korrekt skaliert darstellt.
    \item Pseudo-Noise-Radio - TODO %seriously guys - what is this?
    \item (Video-)Artefakte - Darstellungsfehler, welche sich dadurch zeigen, dass das encodete Video sichtlich stark von dem Referenzvideo abweicht.
\end{itemize}
