\section{Produktleistungen}

\subparagraph{Benutzbarkeit}
\begin{itemize}
	\item \textbf{/L10/} Die Benutzeroberfläche soll einfach und intuitiv zu bedienen sein. Unerfahrene Benutzer sollen ohne Vorwissen in der Lage sein, sich schnell in die Funktionalitäten einzuarbeiten.
	\item \textbf{/L20/} Die Zusammenstellung sowie die Nacheinanderreihung der verschiedenen Manipulationseffekte soll einfach und übersichtlich geschehen können. 
	\item \textbf{/L30/} Der Benutzer muss in der Lage sein, neue Manipulationseffekte hinzuzufügen, ohne die Reihenfolge der bereits bestehenden ändern zu müssen.
	\item \textbf{/L40/} Im Programm sollen theoretisch gleichzeitig uneingeschränkt viele Videos verwaltet, bearbeitet und analysiert werden können.
	\item \textbf{/L50/} Der Benutzer muss ein Video aus jeder Stufe der Bearbeitung problemlos abspielen, analysieren oder exportieren können.
\end{itemize}

\subparagraph{Geschwindigkeit}

\begin{itemize}
	\item \textbf{/L100/} Bei einer einfachen analysefähigen Pipeline mit einem einzigen Manipulationseffekt sollen ein Video der Auflösung 320p und die zugehörigen Statistiken in Echtzeit ausgegeben werden können.
\end{itemize}

\subparagraph{Robustheit}

\begin{itemize}
	\item \textbf{/L200/} Das Programm soll trotz falscher Benutzereingaben oder Parameter nicht abstürzen. Es muss in diesen Fällen entsprechend reagieren.
\end{itemize}

\subparagraph{Erweiterbarkeit}

\begin{itemize}
	\item \textbf{/L300/} Das Programm soll sich einfach um neue Funktionalitäten erweitern lassen.
\end{itemize}

 
