\section{Funktionale Anforderungen}

\subparagraph{Input/Output zur Videobearbeitung} 
\begin{itemize} 
	\item \textbf{/F10/} Einlesen eines Videos im yuv-Format.
	\item \textbf{/F20/} Erzeugung von (Perlin-)Noise.
	\item \textbf{/F30/} Einlesen einer Bilddatei als Standbild-Video.
	\item \textbf{/F40/} Speichern eines Videos als yuv-Datei.
	\item \textbf{/F50/} Speichern und Laden der erstellten Bearbeitungs-/Analysepipelines
	\item \textbf{/F60/} Konstante Farbe als Standbild-Video.
\end{itemize}

\subparagraph{Videobearbeitung}
\begin{itemize}
	\item \textbf{/F100/} Trennung nach den RGB-Farbkanälen.
	\item \textbf{/F110/} Additives Übereinanderlegen mehrerer Videos.
	\item \textbf{/F120/} Gewichtet gemitteltes Übereinanderlegen mehrerer Videos.
	\item \textbf{/F130/} Lineares und Gauß'sches Weichzeichnen von Videos.
	\item \textbf{/F140/} Farbinvertierung eines Videos.
	\item \textbf{/F150/} Quantisieren der Farbpalette eines Videos.
	\item \textbf{/F160/} Änderung von Kontrast, Farbsättigung und Helligkeit eines Videos.
	\item \textbf{/F170/} Verzögern eines Videos um $n$ Frames.	
	\item \textfb{/F180/} Bilden des Differenzvideos zweier Videos.
\end{itemize}

\subparagraph{Videoanalyse}
\begin{itemize}
	\item Analyse eines einzelnen Videos  
		\begin{itemize}
			\item \textbf{/F200/} Anzeigen der I/P/B-Framehäufigkeiten eines Videos.
			\item \textbf{/F210/} Anzeigen des Helligkeitsverlaufs eines Videos pro Frame.
			\item \textbf{/F230/} Anzeigen des Histogramms eines Video.
			\item \textbf{/F240/} Anzeigen der Pseudo-Noies-Ratio pro Frame.
		   \end{itemize} 
	\item Vergleich eines oder mehrerer Videos mit einem Referenzvideo
		\begin{itemize}
			\item \textbf{/F300/} Differenz der Pixelfarben.
			\item \textbf{/F310/} Differenz der Encoder-Entscheidungen.
			\item \textbf{/F320/} Anzahl der Artefakte.
			\item \textbf{/F330/} Videoansicht mit Highlighting der Artefakte.
		\end{itemize}	
	\item Sonstiges
		\begin{itemize}
			\item \textbf{/F400/} Anzeigen eines Videos. 
			\item \textbf{/F410/} Die Punkte /F10/ bis /F330/ sollen im Rahmen der Rechnerleistung beliebig kombinierbar sein.
		\end{itemize}
\end{itemize}

