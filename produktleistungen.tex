\documentclass{article}
\usepackage[utf8]{inputenc}

\begin{document}
\section{Produktleistungen}

\subparagraph{Benutzbarkeit}
\begin{itemize}
\item{\textbf{/L10/} Die Benutzeroberfläche soll einfach und intuitiv zu bedienen sein. Unerfahrene Benutzer sollen ohne Vorwissen in der Lage sein, sich schnell in die Funktionalitäten einzuarbeiten}
%\item{Benutzeroberfläche soll für die schnelle Einarbeitung intuitiv gestaltet sein}
%\item{Unerfahrene Benutzer sollen in der Lage sein, sich schnell in die Funktionalitäten einzuarbeiten}
\item{\textbf{/L20/} Die Zusammenstellung sowie die Nacheinanderreihung der verschiedenen Manipulationseffekte soll einfach und übersichtlich geschehen können}
\item{\textbf{/L30/} Der Benutzer muss in der Lage sein, neue Manipulationseffekte hinzuzufügen, ohne die Reihenfolge der bereits bestehenden ändern zu müssen}
\item{\textbf{/L40/} Im Programm sollen gleichzeitig uneingeschränkt viele Videos verwaltet, bearbeitet und analysiert werden können}
\item{\textbf{/L50/} Der Benutzer muss ein Video aus jeder Stufe der Bearbeitung problemlos abspielen, analysieren oder exportieren können. Dieses Verfahren soll unabhängig von den bis dahin zusammengestellten Filtern/Effekten geschehen}
\item{\textbf{/L60/} Der Benutzer muss in der Lage sein in höchstens $n$ Schritten ein einzelnes Video analysieren zu können}
\end{itemize}

\subparagraph{Geschwindigkeit}

\begin{itemize}
\item{\textbf{/L70/} Es soll möglich sein, mindestens zwei Videodateien ohne sichtbare Antworts-zeitverzögerungen gleichzeitig analysieren zu können}
\item{\textbf{/L80/} Das Programm soll Multicore-Systeme ausnutzen können und bei erhöhter Videoanzahl sowie mehreren Manipulationseffekten entsprechend skalieren}
\item{\textbf{/L90/} Mindestens bei einer einfachen analysefähigen Pipeline mit einem einzigen Manipulationseffekt sollen das veränderte Video und die Statistiken in Echtzeit ausgegeben werden können}
\end{itemize}

\subparagraph{Robustheit}

\begin{itemize}
\item{\textbf{/L100/} Das Programm soll trotz falscher Benutzereingaben oder -parameter nicht abstürzen. Es muss in diesen Fällen entsprechend reagieren}

\end{itemize}

\subparagraph{Erweiterbarkeit}

\begin{itemize}
\item{\textbf{/L110/} Das Programm soll sich einfach um neue Funktionalitäten erwei-tern lassen}
\end{itemize}

\end{document}
 
