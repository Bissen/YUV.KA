\documentclass{article}
\usepackage[latin1]{inputenc}
\usepackage{amsmath}
\usepackage{amssymb}
\begin{document}
\title{Projekt der Softwareentwicklung 2011/2012}
\author{Max Wagner, Patrick Gemander, Sebastian Ullrich, Michael Vollmer \\ Robert Hangu, Daniel Lebert}
\maketitle
\section{Einleitung}

Es soll ein Multimedia-Framework zur Evaluation von Videoencodern erstellt werden. Hierzu sollen Videos erzeugt und ver�ndert werden k�nnen. Diese sollen anschlie�end von einem Videoencoder encodiert werden. Desweiteren sollen die erzeugten Videos mit Referenzvideos verglichen werden, um eventuelle Schw�chen des Encoders aufzuzeigen. In diese Analyse sollen auch Informationen der Encoder einfliessen.

\section{Zielbestimmung}

Die Firma soll durch das Produkt in die Lage versetzt werden, Encoder zu testen, indem encodierte Videos manipuliert und anhand eines Referenzvideos verglichen werden.

\subsection{Musskriterien}

\begin{itemize}
	\item Eingabe von (encodierten) Videos, einer festen Farbe, einem Bild oder Noise
	\item Manipulation der Videos
	\begin{itemize}
		\item Erstellen einer Pipeline mit verschiedenen Manipulationsoptionen
		\item Zusammenstellung �ber knotenbasierte GraphView
	\end{itemize}
	\item Analyse der Videos
	\begin{itemize}
		\item Anzeigen verschiedener Diagramme
		\item Videoausgabe mit optionalen Overlayoptionen
	\end{itemize}
	\item Speichern der Videos und Pipeline
\end{itemize}

\subsection{Wunschkriterien}

\begin{itemize}
	\item Undo/Redo-funktion
	\item Interface mit Tabs
	\item Automatische Resolution-Erkennung
	\item Asynchrone, nichtblockende UI
	\item Multithreading
\end{itemize}

\subsection{Abgrenzungskriterien}

\begin{itemize}
	\item keine Soundausgabe
	\item nur YUV-Videos als Eingabe
	\item keine Speicherung der Graphen und Analysedaten
\end{itemize}

\section{Produkteinsatz}

\begin{itemize}
	\item Das Produkt dient zum Testen der eigens entwickelten Encoder bzw. deren Videos
\end{itemize}

\subsection{Anwendungsbereiche}

\begin{itemize}
	\item Manipulation und Analyse von Videos (wissenschaftlicher Anwendungsbereich)
\end{itemize}

\subsection{Zielgruppen}

\begin{itemize}
	\item Entwickler des zu testenden Encoders
\end{itemize}

\subsection{Betriebsbedingungen}

\begin{itemize}
	\item wissenschaftliche Umgebung
\end{itemize}

\end{document}