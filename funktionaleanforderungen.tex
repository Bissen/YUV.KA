
\documentclass{article}
\usepackage[utf8]{inputenc}
\usepackage{amsmath}
\usepackage{amssymb}
\begin{document}
\title{Funktionale Anforderungen}

\subparagraph{Input/Output zur Videobearbeitung} 
\begin{itemize} 
	\item{\textbf{/F10/} Einlesen eines Videos im yuf-Format.
	\item{\textbf{/F20/} Erzeugung von (Perlin-)Noise.
	\item{\textbg{/F30/} Einlesen einer Bilddatei als Standbild-Video.
	\item{\textbf{/F40/} Speichern eines Videos als yuf-Datei.
	\item{\textbf{/F50/} Speichern und Laden der erstellten Bearbeitungs-/Analysestrukturen
\end{itemize}

\subparagraph{Videobearbeitung}
\begin{itemize}
	\item{\textbf{/F100/} Trennung nach den RGB-Farbkan�len.
	\item{\textbf{/F110/} Additives �bereinanderlegen mehrerer Videos.
	\item{\textbf{/F120/} Gewichtet gemitteltes �bereinanderlegen mehrerer Videos.
	\item{\textbf{/F130/} Lineares und Gau�'sches Weichzeichnen von Videos.
	\item{\textbf{/F140/} Farbinvertierung eines Videos.
	\item{\textbf{/F150/} Quantisieren der Farbpalette eines Videos.
	\item{\textbf{/F160/} �nderung von Kontrast, Farbs�ttigung und Helligkeit eines Videos.
	\item{\textbf{/F170/} Verschiebung eines Videos um $n$ Frames nach Hinten.	
\end{itemize}

\subparagraph{Videoanalyse}
\begin{itemize}
	\item Analyse eines einzelnen Videos  
		\begin{itemize}
			\item{\textbf{/F200/} Anzeigen der I/P/B-Frames eines Videos.
			\item{\textbf{/F210/} Anzeigen des Helligkeitsverlaufs eines Videos pro Frame.
			\item{\textbf{/F220/} Differenz zwischen Frame $n$ und Frame $n + 1$ in einem Video.	
			\item{\textbf{/F230/} Anzeigen des Histogramms eines Video.
			\item{\textbf{/F240/} Anzeigen der Pseodo-Nois-Ratio pro Frame.
		   \end{itemize} 
	\item Vergleich eines oder mehrerer Videos mit einem Referenzvideo
		\begin{itemize}
			\item{\textbf{/F300/} Differenz der Pixelfarben.
			\item{\textbf{/F310/} Differenz der Log-Entscheidungen.
			\item{\textbf{/F320/} Anzahl der Artefakte
			\item{\textbf{/F330/} Videoansicht mit Highlighting der Artefakte.
		\end{itemize}	
	\item Sonstiges
		\begin{itemize}
			\item{\textbf{/F400/} Anzeigen eines Videos. 
			\item{\textbf{/F410/} /F10/ bis /F330/ sollen im Rahmen der Rechnerleistung beliebig kombinierbar sein.
		\end{itemize}
\end{itemize}

\end{document}
